\documentclass[10pt, a4paper]{article}
\usepackage[a4paper, portrait, margin=1in]{geometry}
%\documentclass{article}
\usepackage{graphicx} % Required for inserting images
\usepackage[english, bulgarian]{babel}
\usepackage{amsthm}
\usepackage{amssymb}
\usepackage{array}
\usepackage{hyperref}

\newtheorem{theorem}{Теорема}[section]
\newtheorem{corollary}{Следствие}[theorem]
\newtheorem{lemma}[theorem]{Лема}
\theoremstyle{definition}
\newtheorem{definition}{Дефиниция}[section]
\newtheorem{problem}{Задача}
\newtheorem{prop}{Свойствo}[section]
\newtheorem*{implication}{Малко повече за (би)импликацията}
\theoremstyle{remark}
\newtheorem*{remark}{Забележка}
\newtheorem*{tip}{Полезно}
\newtheorem*{sol}{Решение}



\title{1. Логика}
\date{Юли 2024}

\begin{document}

\maketitle
\section{Преговор}
\begin{definition}
Логически константи - T (true) и F (false)
\end{definition}


\begin{definition}
Прости съждения (логически променливи) - твърдения, които са или истина, или лъжа
\end{definition}
\begin{remark}
Въпросителни, възклицателни, подбудителни изречения, както и такива от вида ''това изречение е лъжа'', неможещи да бъдат нито истина, нито лъжа (защото съдържат противоречие), не са съждения
\end{remark}

\begin{definition}
Съставни съждения - такива, образувани от други съждения и логически константи, посредством логически съюзи
\end{definition}

\begin{definition}[Логически операци]

    \begin{tabular}{ |c|c|c|c|c|c|c|c| }
    \hline
    p & q & \neg p & p \wedge\ q & p \vee\ q & p \oplus\ q & p \rightarrow q & p \leftrightarrow q\\
    \hline
    F & F & T& F& F& F& T& T\\
    \hline
    F& T &T &F &T &T &T &F\\
    \hline
    T&F&F&F&T&T&F&F \\
    \hline
    T&T&F&T&T&F&T&T \\
    \hline
\end{tabular}
\end{definition}

\begin{prop}[Приоритет на логиеските операции]\end{prop}
\begin{enumerate}
\item \emph{негация \(\neg\)} 
\item \emph{конюнкция \(\wedge\)}
\item \emph{изкючващо или \(\oplus\), дизюнкция \(\vee\)}
\item \emph{импликация \(\rightarrow\)}
\item \emph{биимпликация \(\leftrightarrow\)}
\end{enumerate}

\begin{remark}
    Разбира се, при наличие на скоби те са с най-голям приоритет
\end{remark}
\hfill
\begin{implication}
нека p, q са произволни съждения в импликация \(p\rightarrow q\)
\begin{itemize}
    \item p се нарича антецедент, q - консеквент
    \item на импликацията може да се гледа като обещание: нека съм ви дал дума: ''Ако изкарате 100\% на контролното, ще получите оценка 6'' - ако антецедентът е истина (изкарали сте 100\%), то вие ще очаквате да имате 6 (т.е. и консеквентът да е истина), в противен случай обещанието не е изпълнено; ще кажете, че не съм удържал на думата си (т.е. импликацията е лъжа). Разбира се, ако не сте изкарали 100\% (антецедентът е лъжа), няма как да говорим за неспазено обещание, т.е. без значение каква оценка ще получите (независимо консеквента), аз все пак съм казал истината.
    \item антецедентът (p) е свързан с достатъчното условие, а консеквентът с необходимото (q);\\ \emph{Пр. ''Ако съм човек, дишам''} - да си човек е достатъчно, за да твърдим, че дишаш, но не и необходимо (животни и растения също дишат). Обратно, дишането е необходимо условие, за да кажем, че нещо е човек - ако не диша, то не е човек (или в най-добрия случай само е било...), но пък не е достатъчно условие.
    \item импликацията може да се зададе чрез различни езикови конструкции: \emph{\\''ако p, (то) q'', \textbf{но} ''p, \textbf{само} ако q'';\\ ''q (тогава), когато p'', \textbf{но} ''p \textbf{само} (тогава,) когато q'';\\ ''p влече q'', ''q следва от p'', ''p е достатъчно условие за q'', ''q е необходимо условие за p'' } \\
    \begin{remark}
        Забележете, че \emph{''само''} променя смисъла на казаното!
    \end{remark}    
    \item биимпликацията е нещо като двойна импликация (т.е. тук p е и необходимо, и достатъно условие за q, както и обратно), неслучайно отговаря на езиковата конструкция \emph{''тогава и само тогава, когато''}, също и на \emph{''\textbf{точно} тогава, когато''}\\
    \begin{remark}
        Забележете, че \emph{''точно''} променя смисъла на казаното, без него щеше да е просто импликация!
    \end{remark}   
\end{itemize}

\end{implication}

\begin{definition}
    Всеки ред от таблицата на истинност (отговарящ на точно една възможна комбинация от стойности F/T на променливите) наричаме валюация
\end{definition}

\begin{definition}
\hfill
    \begin{itemize}
        \item \emph{тавтология} - съставно съждение, чиято стойност е Т за всяка валюация на просите му съждения
        \item \emph{противоречие} - съставно съждение, чиято стойност е F за всяка валюация
        \item \emph{условност} - съждение, което приема, както стойност T, така и F
    \end{itemize}
\end{definition}

\begin{definition}
    две съждения A и B са еквивалентни (А \( \equiv B\), \(A\Leftrightarrow B\)), тстк съждението \(A\leftrightarrow B\) е тавтология
\end{definition}
\begin{remark}
    A=B би означавало друго - че имат еднаква синтактична структура, т,е. и изглеждат еднакво
\end{remark}
\begin{remark}
    \(\equiv\), \(\Leftrightarrow\) \emph{не} са логически съюзи
\end{remark}
\hfill

\begin{theorem}[еквивалентности]
Нека p, q и r са произволни съждения. Следните еквивалентности са в сила:
    \begin{itemize}
        \item \textbf{свойство на константите:} \(p \vee T \equiv T\), \(p \wedge T \equiv p\), \(p \vee F \equiv p\), \(p \wedge F \equiv F\)
        \item \textbf{свойствa на отрицанието:} \(p\wedge \neg p \equiv F\), \(p\vee \neg p \equiv T\)
        \item \textbf{идемпотентност:} \(p\vee p \equiv p\), \(p\wedge p \equiv p\)
        \item \textbf{закон за двойното отрицание:} \(\neg (\neg p)\equiv p\)
        \item \textbf{комутативност:} \(p\vee q\equiv q\vee p\), \(p\wedge q\equiv q\wedge p\), \(p\oplus q\equiv q\oplus p\)
        \item \textbf{асоциативност:} \((p\vee q)\vee r\equiv p\vee (q\vee r)\), \((p\wedge q)\wedge r\equiv p\wedge (q\wedge r)\), \((p\oplus q)\oplus r\equiv p\oplus (q\oplus r)\)
        \item \textbf{дистрибутичност:} \(p\vee (q\wedge r) \equiv (p\vee q) \wedge (p\vee r)\), \(p\wedge (q\vee r) \equiv (p\wedge q) \vee (p\wedge r)\) 
        \item \textbf{закони на De Morgan:} \(\neg(p\wedge q)\equiv \neg p \vee \neg q\), \(\neg(p\vee q)\equiv \neg p \wedge \neg q\)
        \begin{remark}
            Законите на De Morgan лесно могат да се обобщят за много променливи (как?)
        \end{remark}
        \item \textbf{поглъщане (absorption law):} \(p\vee(p\wedge q)\equiv p\equiv p\wedge(p\vee q)\)
        \item \textbf{Други полезни:}
        \(p \rightarrow q \equiv \neg q \rightarrow \neg p\), \(p \leftrightarrow q \equiv \neg p \leftrightarrow \neg q\), \(p \leftrightarrow q \equiv \neg(p\oplus q)\),\\
        \(p \leftrightarrow q \equiv (p \wedge q) \vee (\neg p \wedge \neg q)\), \(\neg(p\leftrightarrow q)\equiv\ p\leftrightarrow \neg q\),\\
        \(p\vee q \equiv \neg p \rightarrow q\), \(p\wedge q \equiv \neg (p \rightarrow \neg q)\), \(p\wedge \neg q \equiv \neg (p \rightarrow q)\),\\
        \((p \rightarrow q) \wedge (p \rightarrow r) \equiv p \rightarrow (q \wedge r)\), \((p \rightarrow r) \wedge (q \rightarrow r) \equiv (p \vee q) \rightarrow r\),\\ \((p \rightarrow q) \vee (p \rightarrow r) \equiv p \rightarrow (q \vee r)\), \((p \rightarrow r) \vee (q \rightarrow r) \equiv (p \wedge q) \rightarrow r \equiv p \rightarrow (q \rightarrow r)\)
        \begin{remark} На контролни (особено семестриално и изпит) не може да ползвате послендите наготово - изключение правят първите три от тях като по-очевдини и често използвани\end{remark}
    \end{itemize}
\end{theorem}
\hfill

\begin{tip}
\textbf{Доказване на (не)еквивалентност}
    \begin{itemize}
        \item еквивалентност можде да се докаже с:
            \begin{itemize}
                \item таблица на истинност
                \item еквивалентни преобразувания
            \end{itemize}
        \item нееквивалентност можде да се докаже с:
            \begin{itemize}
                \item таблица на истинност
                \item контрапример (подходящ избор на стойности за променливите, за който дадените не са еквивалентни)
            \end{itemize}
    \end{itemize}
\end{tip}

\begin{definition}
    Казваме, че q следва логически от p, ако \(p\rightarrow q\) е тавтология, бележим \(p\vdash q\), също и \(p\Rightarrow q\)
\end{definition}
\begin{remark}\(\vdash/ \Rightarrow\) \emph{не} са логически съюзи, така че изводът \(p\vdash q\) не е съждение, да не се бърка с импликацията!\end{remark}

\begin{definition}[извод в съждителната логика]
    \label{def1.9}
    Извод наричаме последователност от съждения \(p_1, p_2, ...,p_n, q\) /n>0/, където \(p_1\), ... \(p_n\) са предпоставки (premises), а q е следствие (conclusion). Изводът се счита за валиден, когато, \emph{допускайки, че всички предпоставки са верни (т.е. \(p_1\wedge p_2\wedge ... \wedge p_n\equiv T\))}, и следствието е вярно (\(q\equiv T\)). \\Горната дефиниция ни казва, че когато \(p_1\), ... \(p_n\) са едновременно T, искаме и следствието да е T. Това може да се гледа като еквивалентно на това да искаме \(p_1\wedge p_2\wedge ... \wedge p_n\rightarrow q\equiv T\), защото ако някоя от предпоставките е F, то по дефиниция импликацията отново ще е T. В крайна сметка, за да кажем, че изводът е валиден, ще изискваме просто \(p_1\wedge p_2\wedge ... \wedge p_n\rightarrow q\) да е тавтология
\end{definition}


\begin{definition}
    Едноместен предикат е съждение, в което има ''празно място'', в което се слага обект от предварително зададена област, наречена домейн. За всеки обект от домейна, предикатът е или истина, или лъжа. 
\end{definition}
\begin{remark}
    Самият предикат (без да е свързан с обект) още не е съждение, т.е. не е T, нито F
\end{remark}

\begin{definition}[квантори] Често ще ползвате следните (особено по дис):
    \begin{itemize}
        \item \emph{универсален квантор} \(\forall\) - \textbf{за} всяко
        \item \emph{екзистенциален квантор} \(\exists\) - съществува
    \end{itemize}
\end{definition}

\begin{remark} Кванторите имат по-висок приоритет от логическите съюзи \end{remark}

\begin{prop}
\label{prop1.2}
Ако P(x) e предикат над домейн A, състоящ се от обекти \(a_1, ..., a_n\), то:
    \begin{itemize}
        \item \(\exists x\in A: P(x) \equiv P(a_1) \vee P(a_2) \vee \cdots \vee P(a_n)\)    
        \item \(\forall x\in A: P(x) \equiv P(a_1) \wedge P(a_2) \wedge \cdots \wedge P(a_n)\)    
    \end{itemize}
\end{prop}











\section{Основни задачи}
за произволни съждения p, q, r, s:
\begin{problem} Съждения ли са изразите: \(p \equiv q\), \(p\Leftrightarrow q\), \(p\vdash q\), \(p\Rightarrow q\)?
\end{problem}

\begin{problem}
    Ако тази задача е под номер 3, вярно ли е, че \(\oplus\) е символът за конюнкция?
\end{problem}

\begin{problem} Докажете чрез еквивалентни преобразувания закона за поглъщане: \(p\vee(p\wedge q)\equiv p\equiv p\wedge(p\vee q)\) \end{problem}

\begin{problem}
    Докажете чрез еквивалетни преобразувания следните:
    \begin{itemize}
        \item \(p \leftrightarrow q \equiv (p \wedge q) \vee (\neg p \wedge \neg q)\)
        \item \(p \oplus q \equiv (p \vee q) \wedge (\neg p \vee \neg q)\)
        \item \(\neg(p\leftrightarrow q)\equiv\ p\leftrightarrow \neg q\)
    \end{itemize}
\end{problem}

\hfill
\begin{definition}
    Множество от логически операции наричаме функционално затворено/завършено, ако за всяко съждение съществува еквивалетно съждение, съставено само чрез логическите променливи и константи, и въпросните операции. Т.е. всяко съждение можде да се запише, ползвайки само тези операции
\end{definition}

\begin{problem}
Докажете, че множеството от логическите операции \(\neg\), \(\vee\), \(\wedge\) е функционално затворено
\end{problem}

\begin{problem}
Докажете, че множеството от логическите операции \(\neg\), \(\vee\) е функционално затворено. A какво може да се каже за това от операциите \(\neg\), \(\wedge\)?
\end{problem}

\begin{problem}[Семестриално КН 21]  Използвайки еквивалентни преобразувания, докажете следните еквивалентности: 
    \begin{itemize}
        \item \((p \rightarrow q) \wedge (p \rightarrow r) \equiv p \rightarrow (q \wedge r)\)
        \item \((p \rightarrow r) \wedge (q \rightarrow r) \equiv (p \vee q) \rightarrow r\)
        \item \((p \rightarrow q) \vee (p \rightarrow r) \equiv p \rightarrow (q \vee r)\)
        \item \((p \rightarrow r) \vee (q \rightarrow r) \equiv (p \wedge q) \rightarrow r \equiv p \rightarrow (q \rightarrow r)\) 
    \end{itemize}
\end{problem}

\begin{problem}
 Колко предиката ще ползваме, ако разглеждаме твърдението ''Ботев и Вазов са поети'' на езика на предикатната логика?
\end{problem}
\begin{sol}
 Един, идеята е да разбрем, че тук предикатът е ''... е поет'', а просто обектите са два. В крайна сметка, ако домейнът са хората и предикатът P(X) е ''X е поет'', то твърдението ще придобие вида \(P(x) \wedge P(y)\), или в конкретния случай P(Ботев) \(\wedge\) P(Вазов)
\end{sol}

\begin{problem}
    Нека P(x, y), Q(x) са предикати над някакъв домейн. Докажете или опровергайте:
    \begin{itemize}
        \item \(\neg \exists x: Q(x) \equiv \forall x: \neg Q(x)\)
        \item \(\neg \forall x: Q(x) \equiv \exists x: \neg Q(x)\)
        \item \(\forall x\forall y: P(x, y) \equiv \forall x\forall y: P(y, x)\)
        \item \(\exists x\exists y: P(x, y) \equiv \exists x\exists y: P(y, x)\)
        \item \(\forall x\exists y: P(x, y) \vdash \exists x\forall y: P(x, y)\)
        \item \(\exists x\forall y: P(x, y) \vdash \forall x\exists y: P(x, y)\)
    \end{itemize}
\end{problem}

\begin{problem}
Ако P(x), Q(x) са предикати над някакъв домейн, да се докаже, че:
    \begin{itemize}
        \item \(\forall x(P(x)\wedge Q(x)) \equiv \forall x(P(x)) \wedge \forall x(Q(x))\)
        \item \(\exists x(P(x)\vee Q(x)) \equiv \exists x(P(x)) \vee \exists x(Q(x))\)
    \end{itemize}
\end{problem}
\begin{remark}
    Тоест универсалният квантор има дистрибутивно свойство спрямо конюнкцията, а екзистенциалният спрямо дизюнкцията
\end{remark}
\begin{sol}
    Ще разгледаме само първото. Ако \(a_1, \cdots, a_n\) са обектите от домейна, то от \emph{свойство \ref{prop1.2}} \(\forall x(P(x)\wedge Q(x)) \equiv (P(a_1)\wedge Q(a_1)) \wedge \cdots \wedge (P(a_n)\wedge Q(a_n))\), от асоциативността и комутативността на конюнкцията:\\ \((P(a_1)\wedge Q(a_1)) \wedge \cdots \wedge (P(a_n)\wedge Q(a_n)) \equiv\\ (P(a_1) \wedge\cdots\wedge P(a_n)) \wedge (Q(a_1) \wedge\cdots\wedge Q(a_n)) \equiv\\ \forall x(P(x)) \wedge \forall x(Q(x)) \quad\square\) 
\end{sol}

\begin{problem}
Нека P(x, y), Q(x, y), R(x, y) са предикати над някакъв домейн. Напишете отрицанието на следните твърдения така, че знакът за отрицание да не се среща вляво от кванторите
    \begin{itemize}
        \item \(\forall x \exists y ((P(x,y)\wedge Q(x,y))\rightarrow R(x,y))\)
        \item \(\exists x \forall y (P(x,y) \rightarrow (P(x,y)\vee Q(x,y)))\)
    \end{itemize}
\end{problem}



\section{Задачи за подготовка}

\begin{problem}[Семестриално И 24] Ако p, q, r, s, t, x, y и z са съждения, докажете, че изразът:\\ \((p \rightarrow q) \vee (((p \wedge t) \vee (q \wedge x) \vee (r \wedge y)) \rightarrow ((t \rightarrow y) \rightarrow z) \rightarrow p) \vee (q \rightarrow r)\) е тавтология
\end{problem}


\begin{problem}[Семестриално КН 22] Докажете или опровергайте (само чрез еквивалентни преобразувания), че изразът \((\neg p \wedge (p \vee q) \rightarrow q) \rightarrow r\) е тавтология 
\end{problem}

\begin{problem}[Семестриално И 23] Докажете с табличен метод и с еквивалентни преобразувания, че следните са еквивалентни:\\
\(A=\neg( (p \rightarrow q) \wedge (\neg(p \rightarrow r) \vee (\neg q \wedge \neg r)) )\)\\
\(B=(\neg p \wedge q) \vee (p \wedge \neg q) \vee r\)
\end{problem}

\begin{problem}[Семестриално КН 16]
    Вярно ли е, че:
    \begin{itemize}
        \item от \(\forall x(P(x)) \vee \forall x(Q(x))\) следва \(\forall x(P(x)\vee Q(x))\) 
        \item от \(\forall x(P(x)\vee Q(x))\) следва \(\forall x(P(x)) \vee \forall x(Q(x))\) 
    \end{itemize}
\end{problem}

\begin{problem}
    Нека P(x,y) е предикатът ''\(x^2+y^2>2xy\)''. Вярно ли е, че:
    \begin{itemize}
        \item P(-1,2), ако домейнът са всички цели числа
        \item \(\exists x \in \mathbb{N} \exists y \in \mathbb{N}:P(x,y)\)
        \item \(\forall x \in \mathbb{R} \exists y \in \mathbb{N}:P(x,y)\)
        \item \(\forall x \in \mathbb{R^+} \forall y \in \mathbb{R^-}:P(x,y)\)
        \item \(\forall x\) четно \(\exists y\) нечетно: \(\neg P(x,y)\)
        \item \(\exists x\) четно \(\forall y\) нечетно: \(P(x,y)\)
        \item \(\forall x \in \mathbb{R} \forall y \in \mathbb{R}, y>x: P(x,y)\)
        \item \(\forall x \in \mathbb{R} \exists y \in \mathbb{N}:\neg P(x,y)\)
        \item \(\exists x \in \mathbb{R} \exists y \in \mathbb{R}, y\neq x:\neg P(x,y)\)
        \item \(\neg\exists x \in \mathbb{R}, \forall y \in \mathbb{Q}:P(x,y)\)
    \end{itemize}
\end{problem}
\begin{sol}
    Задачата става лесна, след като направим наблюдението, че \(x^2+y^2>2xy\) е същото като \((x-y)^2>0\), което се случва тогава и само тогава, когато \(x\neq y\) (*при реални числа). Ето защо:
    \begin{itemize}
        \item да
        \item да, достатъчно е \(x\neq y\)
        \item да, достатъчно е \(x\neq y\)
        \item да, защото тук винаги \(x\neq y\)
        \item не; уточнихме, че знакът винаги е > (или =), равенство получаваме само при x=y, което е невзъможно, когато са с различна четност
        \item да, всъщност, което и да е четно върши работа
        \item да, защото тук винаги \(x\neq y\)
        \item не, ако x не е естествено, няма как да изберем y=x, така че да ''счупим'' неравенството
        \item не, в началото уточнихме защо
        \item не; ако вземем произволно иррационално число x (т.е. \(x \notin \mathbb{Q}\)), например \(x=\pi\), за кое да е y рационално, \(x\neq y\), а оттук и \((x-y)^2>0\)
        
    \end{itemize}
    
\end{sol}

\hfill
\begin{problem}
    Обяснете защо е същото дали ще имаме извод с предпоставки \(p_1, ... p_n\) и следствие q, или извод с единствена предпоставка \((p_1\wedge ... \wedge p_n)\) и следствие q. 
    Тоест \(p_1\quad \cdots\quad p_n \over \therefore q\) е същото като \((p_1\wedge ... \wedge p_n) \over \therefore q\)
\end{problem}

\begin{problem}[*бонус]
Да се докаже, че изводът с предпоставки \(p_1, ..., p_n\) и следствие \(q\rightarrow r\) е валиден, ако изводът с предпоставки \(p_1, ..., p_n, q\) и следствие r е валиден
\end{problem}
\begin{sol}
Ще покажем два начина (всъщност начинът е един, но формализирането на решението изглежда различно):\\\\
\emph{1 н.)}\\
    Искаме да покажем, че \(p_1\quad \cdots\quad p_n \over \therefore q\to r\).
    По условие имаме, че: \(p_1\quad \cdots\quad p_n\quad q \over \therefore r\), което според \emph{дефиниция \ref{def1.9}} е същото като \((p_1\wedge \cdots\wedge p_n\wedge q)\to r \equiv T\) \emph{(*)} (т.е. е тавтология) . От \emph{(*)}:\\ \(T\equiv\)\\
    \((p_1\wedge \cdots\wedge p_n\wedge q)\to r \equiv\)\\
    \(\neg(p_1\wedge \cdots\wedge p_n\wedge q)\vee r \equiv\)\\
    \((\neg(p_1\wedge \cdots\wedge p_n) \vee \neg q)\vee r \equiv\)\\
    \(\neg(p_1\wedge \cdots\wedge p_n) \vee \neg q\vee r \equiv\)\\
    \(\neg(p_1\wedge \cdots\wedge p_n) \vee (\neg q\vee r) \equiv\)\\
    \(\neg(p_1\wedge \cdots\wedge p_n) \vee (q\to r) \equiv\)\\
    \((p_1\wedge \cdots\wedge p_n) \to (q\to r)\)\\
   Сега според \emph{дефиницията за извод (\ref{def1.9})} \((p_1\wedge \cdots\wedge p_n) \to (q\to r)\equiv T\) ни носи \(p_1\quad \cdots\quad p_n \over \therefore q\to r\), което и искаме. \(\square\)\\\\
\emph{2 н.)}\\
 Тъй като искаме да покажем, че изводът с предпоставки \(p_1, ..., p_n\) и следствие \(q\rightarrow r\) е валиден, то можем да използваме даденото по условие, а именно втория извод (този с предпоставки \(p_1, ..., p_n, q\) и следствие r), за който знаем е валиден, като предпоставка за първия. Тоест искаме:\\
 \begin{align*}
     \(p_1\wedge \cdots\wedge p_n\) /=p/\\
     \((p_1\wedge \cdots\wedge p_n\wedge q)\to r\) /втория извод ползваме като предпоставка/\\
     \(\over \therefore\quad q\to r\)\\
 \end{align*}

\begin{remark}
    За олекотавяне на записа можем да считаме, че \(p_1\wedge \cdots\wedge p_n\) е една голяма предпоставка \(\equiv p\)
\end{remark}
\begin{enumerate}
    \item \(p\wedge q\to r \equiv \neg p\vee \neg q\vee r\) /свойство на импликацията/
    \item \(\neg p\vee \neg q\vee r \equiv \neg p\vee \neg (q\vee r)\)/асоциативност на дизюнкцията/
    \item p (предпоставка)
    \item \(\neg (q\vee r)\) /от 2., 3. и дизюнктивен силогизъм/
    \item \(\neg (q\vee r)\equiv q\to r\) /свойство на импликацията/
    \(\square\)
\end{enumerate}
\end{sol}

\end{document}
