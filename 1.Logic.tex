\documentclass[10pt, a4paper]{article}
\usepackage[a4paper, portrait, margin=1in]{geometry}
%\documentclass{article}
\usepackage{graphicx} % Required for inserting images
\usepackage[english, bulgarian]{babel}
\usepackage{amsthm}
\usepackage{amsmath}
\usepackage{amssymb}
\usepackage{array}
\usepackage{hyperref}
\usepackage[T2A]{fontenc}
\usepackage{hyphenat}

\newtheorem{theorem}{Теорема}[section]
\newtheorem{corollary}{Следствие}[theorem]
\newtheorem{lemma}[theorem]{Лема}
\theoremstyle{definition}
\newtheorem{definition}{Дефиниция}[section]
\newtheorem{problem}{Задача}
\newtheorem{prop}{Свойствo}[section]
\newtheorem*{implication}{Малко повече за (би)импликацията}
\theoremstyle{remark}
\newtheorem*{remark}{Забележка}
\newtheorem*{tip}{Полезно}
\newtheorem*{question}{Въпрос}
\newtheorem*{sol}{Решение}



\title{1. Логика}
\author{''Колега, ми то логично...''}
\date{Октомври 2024}

\begin{document}

\maketitle
\section{Преговор}
\begin{definition}
Логически константи - T (true) и F (false)
\end{definition}


\begin{definition}
Прости съждения (логически променливи) - твърдения, които са или истина, или лъжа
\end{definition}
\begin{remark}
Въпросителни, възклицателни, подбудителни изречения, както и такива от вида ''това изречение е лъжа'', неможещи да бъдат нито истина, нито лъжа (защото съдържат противоречие), не са съждения
\end{remark}

\begin{definition}
Съставни съждения - такива, образувани от други съждения и логически константи, посредством логически съюзи
\end{definition}

\begin{definition}[Логически операци]

    \begin{tabular}{ |c|c|c|c|c|c|c|c| }
    \hline
    p & q & \neg p & p \wedge\ q & p \vee\ q & p \oplus\ q & p \rightarrow q & p \leftrightarrow q\\
    \hline
    F & F & T& F& F& F& T& T\\
    \hline
    F& T &T &F &T &T &T &F\\
    \hline
    T&F&F&F&T&T&F&F \\
    \hline
    T&T&F&T&T&F&T&T \\
    \hline
\end{tabular}
\end{definition}

\begin{question}
     Какъв тогава е резултатът след прилагане на следните операции върху логическите константи: $F\wedge T\vee T$?\\
     \emph{- Отговор}: Всъщност така написаното би имало двояк смисъл ($(F\wedge T)\vee T\equiv T$, но $F\wedge (T\vee T)\equiv F$), ако нямахме приоритет на операциите.
\end{question}
\begin{prop}[Приоритет на логиеските операции]\end{prop}
\begin{enumerate}
\item \emph{негация \(\neg\)} 
\item \emph{конюнкция \(\wedge\)}
\item \emph{изкючващо или \(\oplus\), дизюнкция \(\vee\)}
\item \emph{импликация \(\rightarrow\)}
\item \emph{биимпликация \(\leftrightarrow\)}
\end{enumerate}

\begin{remark}
    Разбира се, при наличие на скоби те са с най-голям приоритет
\end{remark}
\hfill
\begin{implication}
нека p, q са произволни съждения в импликация \(p\rightarrow q\)
\begin{itemize}
    \item p се нарича антецедент, q - консеквент
    \item на импликацията може да се гледа като обещание: нека съм ви дал дума: ''Ако изкарате 100\% на контролното, ще получите оценка 6'' - ако антецедентът е истина (изкарали сте 100\%), то вие ще очаквате да имате 6 (т.е. и консеквентът да е истина), в противен случай обещаното не е изпълнено, ще кажете, че не съм удържал на думата си (т.е. импликацията е лъжа). Разбира се, ако не сте изкарали 100\% (антецедентът е лъжа), няма как да говорим за неспазено обещание, т.е. без значение каква оценка ще получите (независимо консеквента), аз все пак съм казал истината.
    \item антецедентът (p) е свързан с достатъчното условие, а консеквентът с необходимото (q);\\ \emph{Пр. ''Ако съм човек, дишам''} - да си човек е достатъчно, за да твърдим, че дишаш, но не и необходимо (животни и растения също дишат). Обратно, дишането е необходимо условие, за да кажем, че нещо е човек - ако не диша, то не е човек (или в най-добрия случай само е било...), но пък не е достатъчно условие.
    \item импликацията може да се зададе чрез различни езикови конструкции: \emph{\\''ако p, (то) q'', \textbf{но} ''p, \textbf{само} ако q'';\\ ''q (тогава), когато p'', \textbf{но} ''p \textbf{само} (тогава,) когато q'';\\ ''p влече q'', ''q следва от p'', ''p е достатъчно условие за q'', ''q е необходимо условие за p'' } \\
    \begin{remark}
        Забележете, че \emph{''само''} променя смисъла на казаното!
    \end{remark}    
    \item биимпликацията е нещо като двойна импликация (т.е. тук p е и необходимо, и достатъно условие за q, както и обратно), неслучайно отговаря на езиковата конструкция \emph{''тогава и само тогава, когато''}, също и на \emph{''\textbf{точно} тогава, когато''}\\
    \begin{remark}
        Забележете, че \emph{''точно''} променя смисъла на казаното, без него щеше да е просто импликация!
    \end{remark}   
\end{itemize}

\end{implication}

\begin{definition}
    Всеки ред от таблицата на истинност (отговарящ на точно една възможна комбинация от стойности F/T на променливите) наричаме валюация
\end{definition}

\begin{definition}
\hfill
    \begin{itemize}
        \item \emph{тавтология} - съставно съждение, чиято стойност е Т за всяка валюация на просите му съждения
        \item \emph{противоречие} - съставно съждение, чиято стойност е F за всяка валюация
        \item \emph{условност} - съждение, което приема, както стойност T, така и F
    \end{itemize}
\end{definition}

\begin{definition}
    две съждения A и B са еквивалентни (А \( \equiv B\), \(A\Leftrightarrow B\)), тстк съждението \(A\leftrightarrow B\) е тавтология
\end{definition}
\begin{remark}
    A=B би означавало друго - че имат еднаква синтактична структура, т,е. и изглеждат еднакво
\end{remark}
\begin{remark}
    \(\equiv\), \(\Leftrightarrow\) \emph{не} са логически съюзи
\end{remark}
\hfill

\begin{theorem}[еквивалентности]
Нека p, q и r са произволни съждения. Следните еквивалентности са в сила:
    \begin{itemize}
        \item \textbf{свойство на константите:} \(p \vee T \equiv T\), \(p \wedge T \equiv p\), \(p \vee F \equiv p\), \(p \wedge F \equiv F\)
        \item \textbf{свойствa на отрицанието:} \(p\wedge \neg p \equiv F\), \(p\vee \neg p \equiv T\)
        \item \textbf{идемпотентност:} \(p\vee p \equiv p\), \(p\wedge p \equiv p\)
        \item \textbf{закон за двойното отрицание:} \(\neg (\neg p)\equiv p\)
        \item \textbf{комутативност:} \(p\vee q\equiv q\vee p\), \(p\wedge q\equiv q\wedge p\), \(p\oplus q\equiv q\oplus p\)
        \item \textbf{асоциативност:} \((p\vee q)\vee r\equiv p\vee (q\vee r)\), \((p\wedge q)\wedge r\equiv p\wedge (q\wedge r)\), \((p\oplus q)\oplus r\equiv p\oplus (q\oplus r)\)
        \item \textbf{дистрибутичност:} \(p\vee (q\wedge r) \equiv (p\vee q) \wedge (p\vee r)\), \(p\wedge (q\vee r) \equiv (p\wedge q) \vee (p\wedge r)\) 
        \item \textbf{закони на De Morgan:} \(\neg(p\wedge q)\equiv \neg p \vee \neg q\), \(\neg(p\vee q)\equiv \neg p \wedge \neg q\)
        \begin{remark}
            Законите на De Morgan лесно могат да се обобщят за много променливи (как?)
        \end{remark}
        \item \textbf{поглъщане (absorption law):} \(p\vee(p\wedge q)\equiv p\equiv p\wedge(p\vee q)\)
        \item \textbf{свойство на импликацията:} \(p\rightarrow q\equiv \neg p \vee q\)
        \item \textbf{свойство на би-импликацията:} \(p\leftrightarrow q\equiv (p\rightarrow q)\wedge(q\rightarrow p)\)
        \item \textbf{Други полезни:}
        \(p \rightarrow q \equiv \neg q \rightarrow \neg p\), \(p \leftrightarrow q \equiv \neg p \leftrightarrow \neg q\), \(p \leftrightarrow q \equiv \neg(p\oplus q)\),\\
        \(p \leftrightarrow q \equiv (p \wedge q) \vee (\neg p \wedge \neg q)\), \(\neg(p\leftrightarrow q)\equiv\ p\leftrightarrow \neg q\),\\
        \(p\vee q \equiv \neg p \rightarrow q\), \(p\wedge q \equiv \neg (p \rightarrow \neg q)\), \(p\wedge \neg q \equiv \neg (p \rightarrow q)\),\\
        \((p \rightarrow q) \wedge (p \rightarrow r) \equiv p \rightarrow (q \wedge r)\), \((p \rightarrow r) \wedge (q \rightarrow r) \equiv (p \vee q) \rightarrow r\),\\ \((p \rightarrow q) \vee (p \rightarrow r) \equiv p \rightarrow (q \vee r)\), \((p \rightarrow r) \vee (q \rightarrow r) \equiv (p \wedge q) \rightarrow r \equiv p \rightarrow (q \rightarrow r)\)
        \begin{remark} На контролни (особено семестриално и изпит) не може да ползвате послендите наготово - изключение правят първите три от тях като по-очевдини и често използвани\end{remark}
    \end{itemize}
\end{theorem}
\hfill

\begin{tip}
\textbf{Доказване на (не)еквивалентност}
    \begin{itemize}
        \item еквивалентност можде да се докаже с:
            \begin{itemize}
                \item таблица на истинност
                \item еквивалентни преобразувания
            \end{itemize}
        \item нееквивалентност можде да се докаже с:
            \begin{itemize}
                \item таблица на истинност
                \item контрапример (подходящ избор на стойности за променливите, за който дадените не са еквивалентни)
            \end{itemize}
    \end{itemize}
\end{tip}

\begin{definition}
    Казваме, че q следва логически от p, ако \(p\rightarrow q\) е тавтология, бележим \(p\vdash q\), също и \(p\Rightarrow q\)
\end{definition}
\begin{remark}\(\vdash/ \Rightarrow\) \emph{не} са логически съюзи, така че изводът \(p\vdash q\) не е съждение, да не се бърка с импликацията!\end{remark}

\begin{definition}[извод в съждителната логика]
    \label{def1.9}
    Извод наричаме последователност от съждения \(p_1, p_2, ...,p_n, q\) /n>0/, където \(p_1\), ... \(p_n\) са предпоставки (premises), а q е следствие (conclusion). Изводът се счита за валиден, когато, \emph{допускайки, че всички предпоставки са верни (т.е. \(p_1\wedge p_2\wedge ... \wedge p_n\equiv T\))}, и следствието е вярно (\(q\equiv T\)). \\Горната дефиниция ни казва, че когато \(p_1\), ... \(p_n\) са едновременно T, искаме и следствието да е T. Това може да се гледа като еквивалентно на това да искаме \(p_1\wedge p_2\wedge ... \wedge p_n\rightarrow q\equiv T\), защото ако някоя от предпоставките е F, то по дефиниция импликацията отново ще е T. В крайна сметка, за да кажем, че изводът е валиден, ще изискваме просто \(p_1\wedge p_2\wedge ... \wedge p_n\rightarrow q\) да е тавтология
\end{definition}


\begin{definition}
    Едноместен предикат е съждение, в което има ''празно място'', в което се слага обект от предварително зададена област, наречена домейн. За всеки обект от домейна, предикатът е или истина, или лъжа. 
\end{definition}
\begin{remark}
    Самият предикат (без да е свързан с обект) още не е съждение, т.е. не е T, нито F
\end{remark}

\begin{definition}[квантори] Често ще ползвате следните (особено по дис):
    \begin{itemize}
        \item \emph{универсален квантор} \(\forall\) - \textbf{за} всяко
        \item \emph{екзистенциален квантор} \(\exists\) - съществува
    \end{itemize}
\end{definition}

\begin{remark} Кванторите имат по-висок приоритет от логическите съюзи \end{remark}

\begin{prop}
\label{prop1.2}
Ако P(x) e предикат над домейн A, състоящ се от обекти \(a_1, ..., a_n\), то:
    \begin{itemize}
        \item \(\exists x\in A: P(x) \equiv P(a_1) \vee P(a_2) \vee \cdots \vee P(a_n)\)    
        \item \(\forall x\in A: P(x) \equiv P(a_1) \wedge P(a_2) \wedge \cdots \wedge P(a_n)\)    
    \end{itemize}
\end{prop}

\begin{prop}[отрицание и квантори]
    Ако P(x) е предикат над произволен домейн, то:
    \begin{itemize}
        \item \(\neg\forall x: P(x)\equiv \exists x: \neg P(x)\)
        \item \(\neg\exists x: P(x)\equiv \forall x: \neg P(x)\)
    \end{itemize}
\end{prop}










\section{Основни задачи}
за произволни съждения p, q, r, s:
\begin{problem} Съждения ли са изразите: \(p \equiv q\), \(p\Leftrightarrow q\), \(p\vdash q\), \(p\Rightarrow q\)?
\end{problem}
\begin{sol}
    Не, защото \(\equiv,\Leftrightarrow,\vdash,\Rightarrow\) не са логически съюзи. \(\blacksquare\)
\end{sol}

\hfill
\begin{problem}
    Вярно ли е, че ако тази задача е под номер 3, \(\oplus\) е символът за конюнкция?
\end{problem}
\begin{sol}
    Да, вярно е, стига да разгледаме задачата като логическо съждение, в частност импликация с антецедент лъжа (откъдето цялото съждение е истина). \(\blacksquare\)
\end{sol}

\hfill
\begin{problem} Докажете чрез еквивалентни преобразувания закона за поглъщане: \(p\vee(p\wedge q)\equiv p\equiv p\wedge(p\vee q)\) \end{problem}
\begin{sol}
    Доказваме двете равенства поотделно:
    \begin{itemize}
        \item \(p\vee(p\wedge q) \overset{\mathrm{\textup{св-во на константите}}}{\equiv} (p\wedge T)\vee(p\wedge q) \overset{\mathrm{\textup{дистрибутвиност}}}{\equiv} p\wedge(T\vee q) \overset{\mathrm{\textup{св-во на консантите}}}{\equiv} p\wedge T\) \(\overset{\mathrm{\textup{св-во на консантите}}}{\equiv} p\)
        \item \(p\wedge(p\vee q) \overset{\mathrm{\textup{св-во на константите}}}{\equiv} (p\vee F)\wedge(p\vee q) \overset{\mathrm{\textup{дистрибутвиност}}}{\equiv} p\vee(F\wedge q) \overset{\mathrm{\textup{св-во на консантите}}}{\equiv} p\vee F\) \(\overset{\mathrm{\textup{св-во на консантите}}}{\equiv} p\quad\blacksquare\)
    \end{itemize}
\end{sol}

\hfill
\begin{problem}
    Докажете чрез еквивалетни преобразувания следните:
    \begin{itemize}
        \item \(p \leftrightarrow q \equiv (p \wedge q) \vee (\neg p \wedge \neg q)\)
        \item \(p \oplus q \equiv (p \vee q) \wedge (\neg p \vee \neg q)\)
        \item \(\neg(p\leftrightarrow q)\equiv\ p\leftrightarrow \neg q\)
    \end{itemize}
\end{problem}
\begin{sol}\hfill
    \begin{itemize}
        \item \(p \leftrightarrow q \overset{\mathrm{\textup{св-во на би-импликацията}}}{\equiv} (p\rightarrow q)\wedge(q\rightarrow p) \overset{\mathrm{\textup{св-во на импликацията}}}{\equiv} (\neg p \vee q) \wedge (\neg q \vee p) \overset{\mathrm{\textup{дистрибутивност}}}{\equiv} [(\neg p \vee q) \wedge \neg q] \vee [(\neg p \vee q) \wedge p] \overset{\mathrm{\textup{дистрибутивност}}}{\equiv} [(\neg p \wedge \neg q)\vee(q\wedge \neg q)] \vee [(\neg p \wedge p)\vee(q \wedge p)] \overset{\mathrm{\textup{св-ва на отрицанието}}}{\equiv} [(\neg p \wedge \neg q)\vee F] \vee [F\vee(q \wedge p)] \overset{\mathrm{\textup{асоциативност}}}{\equiv} (\neg p \wedge \neg q)\vee F \vee F\vee(q \wedge p) \overset{\mathrm{\textup{св-во на константите}}}{\equiv} (\neg p \wedge \neg q) \vee(p\wedge q)\; \blacksquare\)
        \item \(p\oplus q\equiv \neg(p \leftrightarrow q) \equiv \neg[(\neg p \wedge \neg q)\vee(p\wedge q)] \overset{\mathrm{\textup{De Morgan}}}{\equiv} \neg(\neg p\wedge \neg q) \wedge \neg(p\wedge q) \overset{\mathrm{\textup{De Morgan}}}{\equiv} (p\vee q)\wedge (\neg p\vee \neg q)\;\blacksquare\)
        \item \(\neg(p\leftrightarrow q) \equiv (p\vee q)\wedge (\neg p\vee \neg q) \overset{\mathrm{\textup{комутативност на диз.}}}{\equiv} (q\vee p)\wedge (\neg p\vee \neg q)
        \overset{\mathrm{\textup{св-во на импликацията}}}{\equiv} (\neg q \rightarrow p)\wedge (p\rightarrow \neg q) \overset{\mathrm{\textup{св-во на би-импликацията}}}{\equiv}  p\leftrightarrow \neg q\quad\blacksquare\)
    \end{itemize}
\end{sol}

\hfill
\begin{definition}
    Множество от логически операции наричаме функционално затворено/завършено, ако за всяко съждение съществува еквивалетно съждение, съставено само чрез логическите променливи и константи, и въпросните операции. Т.е. всяко съждение можде да се запише, ползвайки само тези операции
\end{definition}

\begin{problem}
Докажете, че множеството от логическите операции \(\neg\), \(\vee\), \(\wedge\) е функционално затворено
\end{problem}
\begin{sol}
    Достатъчно е да се покаже, че действието на всеки от останалите логически съюзи може да се представи като комбинация на горните три:\\
    \(p\rightarrow q \equiv \neg p\vee q\),\\
    \(p\leftrightarrow q \equiv (p \wedge q) \vee (\neg p \wedge \neg q)\),\\
    \(p\oplus q \equiv (p \vee q) \wedge (\neg p \vee \neg q)\); \(\blacksquare\)
\end{sol}

\hfill
\begin{problem}
Докажете, че множеството от логическите операции \(\neg\), \(\vee\) е функционално затворено. A какво може да се каже за това от операциите \(\neg\), \(\wedge\)?
\end{problem}
\begin{sol}
    Единственото, което е необходимо да направим в добавка на предната задача, е да представим конюнкцията като композиция на негации и дизюнкции. Директно от Де Морган: \(p\wedge q\equiv \neg(\neg p\vee\neg q)\). Действието на останалите съюзи можем да представим чрез негация и дизюнкция, замествайки навсякъде в решението на предната задача конюнкцията с еквивалетното ѝ \(\neg(\neg p\vee\neg q)\). \(\blacksquare\)
\end{sol}

\hfill
\begin{problem}[Семестриално КН 21]  Използвайки еквивалентни преобразувания, докажете следните еквивалентности: 
    \begin{itemize}
        \item \((p \rightarrow q) \wedge (p \rightarrow r) \equiv p \rightarrow (q \wedge r)\)
        \item \((p \rightarrow r) \wedge (q \rightarrow r) \equiv (p \vee q) \rightarrow r\)
        \item \((p \rightarrow q) \vee (p \rightarrow r) \equiv p \rightarrow (q \vee r)\)
        \item \((p \rightarrow r) \vee (q \rightarrow r) \equiv (p \wedge q) \rightarrow r \equiv p \rightarrow (q \rightarrow r)\) 
    \end{itemize}
\end{problem}
\begin{sol}\hfill
    \begin{itemize}
        \item \((p \rightarrow q) \wedge (p \rightarrow r) \overset{\mathrm{\textup{св-во на импликацията}}}{\equiv} (\neg p \vee q) \wedge (\neg p\vee r) \overset{\mathrm{\textup{дистриб.}}}{\equiv} \neg p\vee (q\wedge r) \overset{\mathrm{\textup{св-во на импликацията}}}{\equiv} p\rightarrow (q\wedge r) \quad\blacksquare\)
        \item \((p \rightarrow r) \wedge (q \rightarrow r) \overset{\mathrm{\textup{св-во на импликацията}}}{\equiv} (\neg p\vee r) \wedge (\neg q\vee r) \overset{\mathrm{\textup{дистриб.}}}{\equiv} (\neg p\wedge \neg q) \vee r\overset{\mathrm{\textup{De Morgan}}}{\equiv} \neg(p \vee q) \vee r \overset{\mathrm{\textup{св-во на импликацията}}}{\equiv} (p \vee q) \rightarrow r\quad\blacksquare\)
        \item \((p \rightarrow q) \vee (p \rightarrow r) \overset{\mathrm{\textup{св-во на импликацията}}}{\equiv} (\neg p \vee q) \vee (\neg p\vee r) \overset{\mathrm{\textup{асоциативност на диз.}}}{\equiv} \neg p\vee q\vee\neg p \vee r  \overset{\mathrm{\textup{асоциативност и комутативност}}}{\equiv} (\neg p\vee\neg p)\vee (q\vee r) \overset{\mathrm{\textup{идемпотентност}}}{\equiv} \neg p\vee (q\vee r)
        \overset{\mathrm{\textup{св-во на импликацията}}}{\equiv} p\rightarrow (q\vee r) \quad\blacksquare\)
        \item \((p \rightarrow r) \vee (q \rightarrow r) \overset{\mathrm{\textup{св-во на импликацията}}}{\equiv} (\neg p\vee r) \vee (\neg q\vee r) \overset{\mathrm{\textup{асоциативност на диз.}}}{\equiv} \neg p\vee \neg q\vee r:\)\\
        \\\(\overset{\mathrm{\textup{De Morgan}}}{\equiv} \neg(p\wedge q)\vee r\overset{\mathrm{\textup{св-во на импликацията}}}{\equiv} (p\wedge q)\rightarrow r\quad\square\)\\
        \\\(\overset{\mathrm{\textup{асоциативност на диз.}}}{\equiv} \neg p\vee (\neg q\vee r) \overset{\mathrm{\textup{св-во на импликацията}}}{\equiv} \neg p\vee (q\rightarrow r) \overset{\mathrm{\textup{св-во на импликацията}}}{\equiv} p \rightarrow (q \rightarrow r)\;\blacksquare\)
    \end{itemize}
\end{sol}

\hfill
\begin{problem}
 Колко предиката ще ползваме, ако разглеждаме твърдението ''Ботев и Вазов са поети'' на езика на предикатната логика?
\end{problem}
\begin{sol}
 Един, идеята е да разбрем, че тук предикатът е ''... е поет'', а просто обектите са два. В крайна сметка, ако домейнът са хората и предикатът P(X) е ''X е поет'', то твърдението ще придобие вида \(P(x) \wedge P(y)\), или в конкретния случай P(Ботев) \(\wedge\) P(Вазов) \(\blacksquare\)
\end{sol}

\hfill
\begin{problem}
    Нека P(x, y), Q(x) са предикати над някакъв домейн. Приемаме, че долните съжения са коректно зададени (макар че домейн не е уточнен). Докажете или опровергайте:
    \begin{itemize}
        \item \(\forall x\forall y: P(x, y) \equiv \forall x\forall y: P(y, x)\)
        \item \(\exists x\exists y: P(x, y) \equiv \exists x\exists y: P(y, x)\)
        \item \(\forall x\exists y: P(x, y) \vdash \exists y\forall x: P(x, y)\)
        \item \(\exists x\forall y: P(x, y) \vdash \forall y\exists x: P(x, y)\)
    \end{itemize}
\end{problem}
\begin{sol} Нека \(x_1, ..., x_n\) и \(y_1,...,y_m\) са съответно обектите от       двата домейна.
    \begin{itemize}
        \item  \(\forall x\forall y: P(x, y) \equiv [\forall y P(x_1,y)]\wedge...\wedge[\forall y P(x_n,y)] \equiv [P(x_1,y_1)\wedge ...\wedge P(x_1,y_m)]\wedge...\wedge[P(x_n,y_1)\wedge ...\wedge P(x_n,y_m)] \equiv P(x_1,y_1)\wedge...\wedge P(x_n,y_m) \equiv [P(x_1,y_1)\wedge ...\wedge P(x_n,y_1)]\wedge...\wedge[P(x_1,y_m)\wedge ...\wedge P(x_n,y_m)] \equiv [\forall x P(x,y_1)]\wedge...\wedge[\forall x P(x,y_m)] \equiv \forall y\forall x: P(x,y)\quad \blacksquare\)
        \item \(\exists x\exists y: P(x, y) \equiv \bigvee_{i=1}^n (\exists y P(x_i,y)) \equiv \bigvee_{i=1}^n (\bigvee_{j=1}^m P(x_i,y_j)) \equiv \bigvee_{i=1}^n \bigvee_{j=1}^m P(x_i,y_j) \equiv \bigvee_{j=1}^m \bigvee_{i=1}^n P(x_i,y_j) \equiv \bigvee_{j=1}^m (\bigvee_{i=1}^n P(x_i,y_j)) \equiv \bigvee_{j=1}^m (\exists x P(x,y_j)) \equiv \exists y\exists x: P(x, y) \equiv \exists x\exists y: P(y,x)\quad\blacksquare\)
        \item Не е вярно, ето контрапимер. Нека предикатът P(x,y) е: ''студент \(x\) има факултетен номер \(y\)''. Наистина всеки студент си има факултетен номер: \(\forall x\exists y P(x, y)\). Не е вярно обаче, че съществува номер, който е факултетен едновременно за всички студенти (\(\exists y\forall x: P(x, y)\))\(\quad\blacksquare\)
        \item Изводът е валиден, защото: за някое \(x_0,\forall y: P(x_0, y) \Rightarrow \forall y \exists x=x_0: P(x_0,y);\quad\blacksquare\)\\
        \\Ако търсим формалност, можем да докажем друго, че \(\exists x\forall y P(x, y) \rightarrow \forall y\exists x P(x, y)\) е тавтология: \(\exists x\forall y P(x, y) \rightarrow \forall y\exists x P(x, y) \equiv \neg\exists x\forall y P(x, y) \vee \forall y\exists x P(x, y)\equiv \forall x\exists y \neg P(x, y) \vee \forall y\exists x P(x, y) \equiv \bigwedge_{i=1}^n[\bigvee_{j=1}^m\neg P(x_i,y_j)]\vee \bigwedge_{k=1}^m[\bigvee_{l=1}^n P(x_l,y_k)] \overset{\mathrm{\textup{дистриб.}}}{\equiv} \bigwedge_{i,k=1}^{n,m} [[\bigvee_{j=1}^m\neg P(x_i,y_j)] \vee [\bigvee_{l=1}^n P(x_l,y_k)]] \overset{\mathrm{\textup{асоциативност на диз.}}}\equiv \bigwedge_{i,k=1}^{n,m} [\bigvee_{j=1}^m\neg P(x_i,y_j) \vee \bigvee_{l=1}^n P(x_l,y_k)] \overset{\mathrm{\textup{разглеждаме j=k, l=i}}}\equiv
        \bigwedge_{i,k=1}^{n,m} [... \vee \neg P(x_i,y_k) \; ... \vee P(x_i,y_k)\;...] \equiv \bigwedge_{i,k=1}^{n,m}[T]\equiv T\quad\blacksquare\)

    \end{itemize}
    \begin{remark}
        Макар че формалният запис е \emph{изключително затормозяващ и нечетим}, горните преобразувания всъщност са доста прости откъм идея. (Не е необходимо да ги четете /аз не бих/, достатъчно е да можете сами да ''облечете'' идеите си в подобен запис.) 
    \end{remark}
\end{sol}

\hfill
\begin{problem}
Ако P(x), Q(x) са предикати над някакъв домейн, да се докаже, че:
    \begin{itemize}
        \item \(\forall x(P(x)\wedge Q(x)) \equiv \forall x(P(x)) \wedge \forall x(Q(x))\)
        \item \(\exists x(P(x)\vee Q(x)) \equiv \exists x(P(x)) \vee \exists x(Q(x))\)
    \end{itemize}
\end{problem}
\begin{remark}
    Тоест универсалният квантор има дистрибутивно свойство спрямо конюнкцията, а екзистенциалният спрямо дизюнкцията
\end{remark}
\begin{sol}
    Ще разгледаме само първото. Ако \(a_1, \cdots, a_n\) са обектите от домейна, то от \emph{свойство \ref{prop1.2}} \(\forall x(P(x)\wedge Q(x)) \equiv (P(a_1)\wedge Q(a_1)) \wedge \cdots \wedge (P(a_n)\wedge Q(a_n))\), от асоциативността и комутативността на конюнкцията:\\ \((P(a_1)\wedge Q(a_1)) \wedge \cdots \wedge (P(a_n)\wedge Q(a_n)) \equiv\\ (P(a_1) \wedge\cdots\wedge P(a_n)) \wedge (Q(a_1) \wedge\cdots\wedge Q(a_n)) \equiv\\ \forall x(P(x)) \wedge \forall x(Q(x)) \quad\blacksquare\) 
\end{sol}

\hfill
\begin{problem}
Нека P(x, y), Q(x, y), R(x, y) са предикати над някакви домейни. Напишете отрицанието на следните твърдения така, че знакът за отрицание да не се среща вляво от кванторите
    \begin{itemize}
        \item \(\forall x \exists y [(P(x,y)\wedge Q(x,y))\rightarrow R(x,y)]\)
        \item \(\exists x \forall y [P(x,y) \rightarrow (P(x,y)\vee Q(x,y))]\)
    \end{itemize}
\end{problem}
\begin{sol}\hfill
    \begin{itemize}
        \item \(\neg\forall x \exists y [(P(x,y)\wedge Q(x,y))\rightarrow R(x,y)] \equiv \exists x \forall y \neg[(P(x,y)\wedge Q(x,y))\rightarrow R(x,y)] \overset{\mathrm{\textup{св-во на импликацията}}}{\equiv} \exists x \forall y \neg[\neg(P(x,y)\wedge Q(x,y))\vee R(x,y)] \overset{\mathrm{\textup{De Morgan}}}{\equiv} \exists x \forall y \neg[(\neg P(x,y)\vee \neg Q(x,y))\vee R(x,y)] \overset{\mathrm{\textup{асоциативност}}}{\equiv} \exists x \forall y \neg[\neg P(x,y)\vee \neg Q(x,y)\vee R(x,y)] \overset{\mathrm{\textup{De Morgan}}}{\equiv} \exists x \forall y[P(x,y)\wedge Q(x,y)\wedge \neg R(x,y)]\quad \blacksquare\) 
        \item \(\neg\exists x \forall y [P(x,y) \rightarrow (P(x,y)\vee Q(x,y))] \equiv \forall x\exists y\neg[P(x,y) \rightarrow (P(x,y)\vee Q(x,y))] \overset{\mathrm{\textup{св-во на импликацията}}}{\equiv} \forall x\exists y \neg[\neg P(x,y) \vee (P(x,y)\vee Q(x,y))] \overset{\mathrm{\textup{асоциативност}}}{\equiv} \forall x\exists y \neg[\neg P(x,y) \vee P(x,y)\vee Q(x,y)]  \overset{\mathrm{\textup{De Morgan}}}{\equiv} \forall x\exists y [P(x,y) \wedge \neg P(x,y)\wedge \neg Q(x,y)]\equiv F\quad \blacksquare\)
    \end{itemize}
\end{sol}


\section{Задачи за подготовка}

\begin{problem}[Семестриално И 24] Ако p, q, r, s, t, x, y и z са съждения, докажете, че изразът:\\ \((p \rightarrow q) \vee [((p \wedge t) \vee (q \wedge x) \vee (r \wedge y)) \rightarrow ((t \rightarrow y) \rightarrow z) \rightarrow p] \vee (q \rightarrow r)\) е тавтология
\end{problem}
\begin{sol}
    От комутативността на дизюнкцията даденото е същото като: \((p \rightarrow q)\vee (q \rightarrow r) \vee [((p \wedge t) \vee (q \wedge x) \vee (r \wedge y)) \rightarrow ((t \rightarrow y) \rightarrow z) \rightarrow p]\equiv [(\neg p\vee q)\vee (\neg q\vee r)]\vee [...] \overset{\mathrm{\textup{асоциативност}}}{\equiv} [\neg p\vee q\vee \neg q\vee r]\vee[...] \overset{\mathrm{\textup{св-во на отрицанието}}}{\equiv} [\neg p\vee T\vee r]\vee[...]\equiv  T\quad\blacksquare\)
\end{sol}

\hfill
\begin{problem}[Семестриално КН 22] Докажете или опровергайте (само чрез еквивалентни преобразувания), че изразът \((\neg p \wedge (p \vee q) \rightarrow q) \rightarrow r\) е тавтология 
\end{problem}
\begin{sol}
    \((\neg p \wedge (p \vee q) \rightarrow q) \rightarrow r \overset{\mathrm{\textup{св-во на импликацията}}}{\equiv} 
    (\neg(\neg p \wedge (p \vee q))\vee q)\rightarrow r
    \overset{\mathrm{\textup{De Morgan}}}{\equiv}
    ((p\vee\neg(p\vee q))\vee q)\rightarrow r
    \overset{\mathrm{\textup{асоциативност}}}{\equiv}
    (p\vee\neg(p\vee q)\vee q)\rightarrow r \overset{\mathrm{\textup{комутат. и асоциат.}}}{\equiv} ((p\vee q)\vee \neg(p\vee q))\rightarrow r\equiv T\rightarrow r\equiv r
    \) значи е достатъчно да изберем \(r\equiv F\), за да бъде цялото съждение грешно, т.е. не е тавтология.\\
    Можем и направо да дадем контрапример, полагайки \(p\equiv q\equiv r\equiv F\quad\blacksquare\).
\end{sol}

\hfill
\begin{problem}[Семестриално И 21]
Нека p, q и r са произволни съждения. Докажете чрез еквивалентни преобразувания, че:
\begin{itemize}
    \item \((p\wedge q)\vee(p\wedge q\wedge r)\equiv p\wedge q\)
    \item \((p\vee q)\wedge(p\vee q\vee r)\equiv p\vee q\)
\end{itemize}
\end{problem}
\begin{sol}
    Задачата се решава доста елегантно, ако положим \(s\equiv(p\wedge q), t\equiv(p\vee q)\), тогава:
    \begin{itemize}
        \item \((p\wedge q)\vee(p\wedge q\wedge r)\equiv (p\wedge q)\vee((p\wedge q)\wedge r) \equiv s\vee(s\wedge r) \equiv s\) директно от закона за поглъщане. \(\blacksquare\)
        \item \((p\vee q)\wedge(p\vee q\vee r)\equiv (p\vee q)\wedge((p\vee q)\vee r) \equiv t\wedge(t\vee r) \equiv t\) директно от закона за поглъщане. \(\blacksquare\)
    \end{itemize}
\end{sol}

\hfill
\begin{problem}[Семестриално И 23] Докажете с табличен метод и с еквивалентни преобразувания, че следните са еквивалентни:\\
\(A=\neg( (p \rightarrow q) \wedge (\neg(p \rightarrow r) \vee (\neg q \wedge \neg r)) )\)\\
\(B=(\neg p \wedge q) \vee (p \wedge \neg q) \vee r\)
\end{problem}
\begin{sol}
    \(A=\neg( (p \rightarrow q) \wedge (\neg(p \rightarrow r) \vee (\neg q \wedge \neg r)))\overset{\mathrm{\textup{св-во на импликацията}}}\equiv\\
    \neg( (\neg p \vee q) \wedge (\neg(\neg p \vee r) \vee (\neg q \wedge \neg r)))\overset{\mathrm{\textup{De Morgan}}}\equiv\\
    \neg( (\neg p \vee q) \wedge ((p \wedge \neg r) \vee (\neg q \wedge \neg r)))\overset{\mathrm{\textup{De Morgan}}}\equiv\\
    \neg(\neg p \vee q) \vee \neg((p \wedge \neg r) \vee (\neg q \wedge \neg r))\overset{\mathrm{\textup{De Morgan}}}\equiv\\
    (p \wedge \neg q) \vee (\neg(p \wedge \neg r) \wedge \neg(\neg q \wedge \neg r))\overset{\mathrm{\textup{De Morgan}}}\equiv\\
    (p \wedge \neg q) \vee ((\neg p \vee r) \wedge (q \vee r))\overset{\mathrm{\textup{дистрибутивност}}}\equiv\\
    (p \wedge \neg q) \vee (((\neg p \vee r) \wedge q) \vee ((\neg p \vee r) \wedge r)))\overset{\mathrm{\textup{поглъщане}}}{\equiv}\\
    (p \wedge \neg q) \vee (((\neg p\wedge q) \vee (r \wedge q) \vee r))\overset{\mathrm{\textup{поглъщане}}}\equiv\\
    (p \wedge \neg q) \vee (((\neg p\wedge q) \vee r))\overset{\mathrm{\textup{асоциативност и комутативност}}}\equiv\\
    (\neg p \wedge q) \vee (p \wedge \neg q) \vee r\equiv B\quad\blacksquare\)
\end{sol}

\hfill
\begin{problem}[Семестриално КН 16]
    Вярно ли е, че:
    \begin{itemize}
        \item от \(\forall x(P(x)) \vee \forall x(Q(x))\) следва \(\forall x(P(x)\vee Q(x))\) 
        \item от \(\forall x(P(x)\vee Q(x))\) следва \(\forall x(P(x)) \vee \forall x(Q(x))\) 
    \end{itemize}
\end{problem}
\begin{sol}\hfill
    \begin{itemize}
        \item За да бъде \(\forall x(P(x)) \vee \forall x(Q(x))\equiv T\), то поне един от двата операнда на дизюнкцията е истина, б.о.о \(\forall x(P(x))\equiv T\Rightarrow \forall x(P(x)\vee Q(x))\equiv T.\quad\blacksquare\)
        \item Не, не следва. Например, ако предикатът P(x) е: ''\(x\) има брат'', а предикатът Q(x): ''\(x\) има сестра'' и знаем, че всеки \(x\) от домейна има брат или сестра, \(\forall x(P(x)\vee Q(x))\), но оттук не следва, че всички имат брат или всички имат сестра, т.е \(\forall x(P(x))\vee\forall x(Q(x)).\quad\blacksquare\)
    \end{itemize}
\end{sol}

\hfill
\begin{problem}
    Нека P(x,y) е предикатът ''\(x^2+y^2>2xy\)''. Вярно ли е, че:
    \begin{itemize}
        \item P(-1,2), ако домейнът са всички цели числа
        \item \(\exists x \in \mathbb{N} \exists y \in \mathbb{N}:P(x,y)\)
        \item \(\forall x \in \mathbb{R} \exists y \in \mathbb{N}:P(x,y)\)
        \item \(\forall x \in \mathbb{R^+} \forall y \in \mathbb{R^-}:P(x,y)\)
        \item \(\forall x\) четно \(\exists y\) нечетно: \(\neg P(x,y)\)
        \item \(\exists x\) четно \(\forall y\) нечетно: \(P(x,y)\)
        \item \(\forall x \in \mathbb{R} \forall y \in \mathbb{R}, y>x: P(x,y)\)
        \item \(\forall x \in \mathbb{R} \exists y \in \mathbb{N}:\neg P(x,y)\)
        \item \(\exists x \in \mathbb{R} \exists y \in \mathbb{R}, y\neq x:\neg P(x,y)\)
        \item \(\neg\exists x \in \mathbb{R}, \forall y \in \mathbb{Q}:P(x,y)\)
    \end{itemize}
\end{problem}
\begin{sol}
    Задачата става лесна, след като направим наблюдението, че \(x^2+y^2>2xy\) е същото като \((x-y)^2>0\), което се случва тогава и само тогава, когато \(x\neq y\) (*при реални числа). Ето защо:
    \begin{itemize}
        \item да
        \item да, достатъчно е \(x\neq y\)
        \item да, достатъчно е \(x\neq y\)
        \item да, защото тук винаги \(x\neq y\)
        \item не; уточнихме, че знакът винаги е > (или =), равенство получаваме само при x=y, което е невзъможно, когато са с различна четност
        \item да, всъщност, което и да е четно върши работа
        \item да, защото тук винаги \(x\neq y\)
        \item не, ако x не е естествено, няма как да изберем y=x, така че да ''счупим'' неравенството
        \item не, в началото уточнихме защо
        \item не; ако вземем произволно иррационално число x (т.е. \(x \notin \mathbb{Q}\)), например \(x=\pi\), за кое да е y рационално, \(x\neq y\), а оттук и \((x-y)^2>0\quad\blacksquare\)
        
    \end{itemize}
    
\end{sol}

\hfill
\begin{problem}
    Обяснете защо е същото дали ще имаме извод с предпоставки \(p_1, ... p_n\) и следствие q, или извод с единствена предпоставка \((p_1\wedge ... \wedge p_n)\) и следствие q. 
    Тоест \(p_1\quad \cdots\quad p_n \over \therefore q\) е същото като \((p_1\wedge ... \wedge p_n) \over \therefore q\)
\end{problem}
\begin{sol}
    Извод с предпоставки \(p_1,...,p_n\) и следствие \(q\) е валиден точно когато \(p_1\wedge....\wedge p_n\rightarrow q\equiv T\) е тавтология. Извод с единствена предпоставка \((p_1\wedge ... \wedge p_n)\) и следствие \(q\) пък е валиден точно когато \((p_1\wedge....\wedge p_n)\rightarrow q\equiv T\) е тавтология, което е същото като горното. Тоест двата извода, имащи еднакво следствие, са еквивалентни (единият е верен точно когато и другият е). \(\quad\blacksquare\)
\end{sol}

\hfill
\begin{problem}[\#бонус]
Да се докаже, че изводът с предпоставки \(p_1, ..., p_n\) и следствие \(q\rightarrow r\) е валиден, ако изводът с предпоставки \(p_1, ..., p_n, q\) и следствие r е валиден
\end{problem}
\begin{sol}
Ще покажем два начина (всъщност начинът е един, но формализирането на решението изглежда различно):\\\\
\emph{1 н.)}\\
    Искаме да покажем, че \(p_1\quad \cdots\quad p_n \over \therefore q\to r\).
    По условие имаме, че: \(p_1\quad \cdots\quad p_n\quad q \over \therefore r\), което според \emph{дефиниция \ref{def1.9}} е същото като \((p_1\wedge \cdots\wedge p_n\wedge q)\to r \equiv T\) \emph{(*)} (т.е. е тавтология) . От \emph{(*)}:\\ \(T\equiv\)\\
    \((p_1\wedge \cdots\wedge p_n\wedge q)\to r \equiv\)\\
    \(\neg(p_1\wedge \cdots\wedge p_n\wedge q)\vee r \equiv\)\\
    \((\neg(p_1\wedge \cdots\wedge p_n) \vee \neg q)\vee r \equiv\)\\
    \(\neg(p_1\wedge \cdots\wedge p_n) \vee \neg q\vee r \equiv\)\\
    \(\neg(p_1\wedge \cdots\wedge p_n) \vee (\neg q\vee r) \equiv\)\\
    \(\neg(p_1\wedge \cdots\wedge p_n) \vee (q\to r) \equiv\)\\
    \((p_1\wedge \cdots\wedge p_n) \to (q\to r)\)\\
   Сега според \emph{дефиницията за извод (\ref{def1.9})} \((p_1\wedge \cdots\wedge p_n) \to (q\to r)\equiv T\) ни носи \(p_1\quad \cdots\quad p_n \over \therefore q\to r\), което и искаме. \(\blacksquare\)\\\\
\emph{2 н.)}\\
 Тъй като искаме да покажем, че изводът с предпоставки \(p_1, ..., p_n\) и следствие \(q\rightarrow r\) е валиден, то можем да използваме даденото по условие, а именно втория извод (този с предпоставки \(p_1, ..., p_n, q\) и следствие r), за който знаем е валиден, като предпоставка за първия. Тоест искаме:
     \[p_1\wedge \cdots\wedge p_n (=p)\]
     \[(p_1\wedge \cdots\wedge p_n\wedge q)\to r \textup{/втория извод ползваме като предпоставка/} \over \therefore\quad q\to r\]\\

\begin{remark}
    За олекотавяне на записа можем да считаме, че \(p_1\wedge \cdots\wedge p_n\) е една голяма предпоставка \(\equiv p\)
\end{remark}
\begin{enumerate}
    \item \(p\wedge q\to r \equiv \neg p\vee \neg q\vee r\) /свойство на импликацията/
    \item \(\neg p\vee \neg q\vee r \equiv \neg p\vee \neg (q\vee r)\)/асоциативност на дизюнкцията/
    \item p (предпоставка)
    \item \(\neg (q\vee r)\) /от 2., 3. и дизюнктивен силогизъм/
    \item \(\neg (q\vee r)\equiv q\to r\) /свойство на импликацията/
    \(\blacksquare\)
\end{enumerate}
\end{sol}

\end{document}
